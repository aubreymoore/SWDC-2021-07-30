%\documentclass[show notes]{beamer}
%\documentclass[handout]{beamer}
\documentclass[]{beamer}

\usepackage{pgfpages}
\usepackage[utf8]{inputenc}
\usepackage[T1]{fontenc}
\usepackage{mathabx}
\usepackage{mathpazo}
\usepackage{eulervm}
\usepackage{natbib}
\usepackage{adjustbox}
\usepackage{booktabs}

\usetheme{Madrid}
\definecolor{uog}{rgb}{0,.5,0}
\usecolortheme[named=uog]{structure}

\mode<handout>{
    \pgfpagesuselayout{4 on 1}[letterpaper] 
    \setbeameroption{show notes}
}


% The following code uses \AtBeginSection to place a frame with the section title (\insertsectionhead) inside a beamercolorbox.
% From https://tex.stackexchange.com/questions/178800/creating-sections-each-with-title-pages-in-beamers-slides
\AtBeginSection[]{
  \begin{frame}
  \vfill
  \centering
  \begin{beamercolorbox}[sep=8pt,center,shadow=true,rounded=true]{title}
    \usebeamerfont{title}\insertsectionhead\par%
  \end{beamercolorbox}
  \vfill
  \end{frame}
}

\title[Biological Invasion of Guam]{Biological Invasion of Guam}

\author{Aubrey Moore}

\institute[University of Guam]{Cooperative Extension Service\\College of Natural and Applied Sciences\\University of Guam}

\titlegraphic{\includegraphics[width=2cm]{big_g2.pdf}}

\date[]{WEDA/WAAESD Joint Summer Meeting, Guam\\July 11, 2018}

\begin{document}

\maketitle

\begin{frame}{Outline}
    \tableofcontents
\end{frame}

\section{What is a biodiversity inventory and why do we need one?}

\begin{frame}{What is a biodiversity inventory?}
A biodiversity inventory is essentially a database containing a comprehensive check list of all taxa known occur within a defined area.
\end{frame}

\begin{frame}{Why do we need a biodiversity inventory?}
  \begin{itemize}
      \item To document rapid changes to Guam's ecosystems
      \item To provide free, open access to information on Guam's flora and fauna
      \item To share Guam biodiversity information with the global scientific community, policy makers and the public
  \end{itemize}
\end{frame}

\begin{frame}{Why do we need a biodiversity inventory?}
The Guam Biodiversity Inventory will facilitate automatic generation and updates to lists such as:
\begin{itemize}
\item A list of all invasive species on Guam with year first recorded
\item A list of new species described from specimens collected on Guam 
\item A list of observations for Guam's endangered species
\item A list of Guam's native plants with associated herbivores and pathogens
\item A list of crops grown on Guam and pests and pathogens which attack them
\item A list of pests and associated biological control agents
\item For any taxon, a literature reference list and links to images
\item Taxonomic checklists and field guides with images
\end{itemize}
\end{frame}

\section{Status of Guam's biodiversity}

\begin{frame}{Status of Guam's biodiversity}
    Guam's biodiversity is being threatened by a HIPPO!
    \adjincludegraphics[height=0.85\textheight,center]{hippo.jpg}
\end{frame}

\begin{frame}{Status of Guam's biodiversity}
    Guam's biodiversity is being threatened by a HIPPO!
    \begin{description}[align=left]
        \item [\texttt{H}] Habitat loss
        \item [\texttt{I}] Invasive Species
        \item [\texttt{P}] Pollution 
        \item [\texttt{P}] Human Population
        \item [\texttt{O}] Overharvesting
    \end{description}
\end{frame}

\begin{frame}{Status of Guam's biodiversity}
Guam's ecosystems are heavily impacted by invasive species
\begin{itemize}
\item Most inland bird species on Guam became extinct or extirpated in the wild after arrival of the brown treesnake
\item 90\%  of Guam's endemic fadang plants, \textit{Cycas micronesica}, have been killed by Asian cycad scale and other invasive species
\item Guam's coconut palms and other palms are rapidly being killed by coconut rhinoceros beetle
\item Estimated arrival rate for invasive species never before recorded on Guam is 1 species per day
\end{itemize}
\end{frame}

\begin{frame}{Bird extinction by brown treesnake}
	\adjincludegraphics[height=0.85\textheight,center]{bts.jpg}
    \tiny{Courtesy of USGS}
\end{frame}

\begin{frame}{Massive mortality of \textit{Cycas micronesica} by invasive species}
	\note[item]{Invasive species have killed about 90\% of Guam's endemic \textit{C. micronesica} plants and the population is not recovering}
	\note[item]{\textit{C. micronesica} went from being the most numerous tree in Guam's forests in 2002 to being placed on the National Endangered Species list in 2016}
	\adjincludegraphics[height=0.9\textheight,center]{CAS.png}
\end{frame}

\begin{frame}{Massive mortality of palm trees by coconut rhinoceros beetle}
	\note[item]{ Many coconut palms and other palms are being killed by an uncontrolled outbreak of coconut rhinoceros beetle}
	\note[item]{Coconut palm was the second most abundant tree in Guam's forests in 2002}
	\begin{figure}
		\begin{minipage}[t]{.48\textwidth}
			\adjincludegraphics[valign=t,width=\textwidth]{rhino_beetle_head_r.jpg}
            \tiny{Courtesy of US Forest Service}
		\end{minipage}
		\begin{minipage}[t]{0.48\textwidth}
			\adjincludegraphics[valign=t,width=\textwidth]{dying_coconuts.jpg}
		\end{minipage}
	\end{figure} 
\end{frame}


\begin{frame}{Impediments in studying Guam's biodiversity}
   
    \begin{block}{Endemicity}
        Many taxa on Guam are endemic and many (most?) of these have not been described.
    \end{block}
    
    \begin{block}{Taxonomic impediment}
    	It is hard to find taxonomists with specialized knowledge and skills to identify many groups of organisms on Guam (especially insects). 
    \end{block}
    
\end{frame}{}


\begin{frame}{Biodiversity survey examples}

    \begin{block}{DoD Baseline Survey of Terrestrial Arthropods on Pagan 2010}
     288 taxa; 228 new island records; 4 new (undescribed) species
    \end{block}
    
    \begin{block}{DoD Baseline Survey of Terrestrial Arthropods on Guam 2012}
    9,099 specimens collected; identified to taxonomic rank of Order only!!
    \end{block}
    
    \begin{block}{Bark Beetle (Scolytinae) Trapping on Guam 2011}
    A single trap at a single location was operated for 2 months.\\
    7 species; 3 new island records
    \end{block}
    
    \begin{block}{Survey of Drosophilidae on Guam 2012}
    24 species; 14 new island records; 5 new (undescribed) species
    \end{block}
    
\end{frame}


\section{Biodiversity data sources for Guam}

\begin{frame}{What data are stored in a biodiversity inventory?}
\begin{itemize}
\item \textbf{Taxonomic data:} A hierarchical checklist containing taxon names, synonyms, vernacular names, and references to publications containing taxon descriptions
\item \textbf{Occurrence data:} Specimen data including date of collection, location, images, ecological associations, DNA sequences, morphometrics, etc.
\item \textbf{Literature data:} References to information on taxa published in journals and reports
\end{itemize}
\end{frame}

\begin{frame}{Biological collections as a biodiversity data source}

	\begin{figure}
		\begin{minipage}[t]{.48\textheight}
			\adjincludegraphics[valign=t,width=\textwidth]{bug.jpg}
		\end{minipage}
		\begin{minipage}[t]{0.25\textheight}
			\adjincludegraphics[valign=t,width=\textwidth]{herbarium_sheet.jpg}
		\end{minipage}
	\end{figure} 

% Table created using https://www.tablesgenerator.com/
\begin{table}[]
\centering
%\caption{My caption}
\label{my-label}
\begin{tabular}{@{}ll@{}}
\toprule
Collection            & Digital Catalog                      \\ \midrule
UOG Insect Collection & Biota; BioLink                       \\
UOG Herbarium         & Borland Reflex; FileMaker; MS Access \\
UoG Fish Collection   & Specify                              \\
UoG Diatom Collection & Specify                              \\ \bottomrule
\end{tabular}
\end{table}

\end{frame}

%\end{markdown}

\section{Biodiversity data standards and tools}

\begin{frame}{Biodiversity data standards and tools}
	\adjincludegraphics[width=\textwidth,center]{diag2.pdf}    
\end{frame}


\begin{frame}{GBIF home}
	\adjincludegraphics[width=\textwidth,center]{gbif_home.png}    
\end{frame}


\begin{frame}{GBIF occurrences map}
	\adjincludegraphics[width=\textwidth,center]{gbif_guam_occurrences_map.png}   
\end{frame}


\begin{frame}{GBIF occurrences table}
	\adjincludegraphics[width=\textwidth,center]{gbif_guam_occurrences_table.png}  
\end{frame}


\begin{frame}{GBIF occurrences datasets}
	\adjincludegraphics[width=\textwidth,center]{gbif_guam_occurrences_datasets.png}    
\end{frame}


\begin{frame}{GBIF occurrences gallery}
	\adjincludegraphics[width=\textwidth,center]{gbif_guam_occurrences_gallery.jpg}    
\end{frame}


\begin{frame}{GBIF simple search for "Guam"}
	\adjincludegraphics[height=.8\textheight,center]{gbif_guam_everything.png}    
\end{frame}


\begin{frame}{GBIF Backbone Taxonomy}
The GBIF Backbone Taxonomy:
\begin{itemize}
\item is the "Tree of Life" used by GBIF to deal with all taxon names within its database
\item currently includes 2.7 million accepted taxon names and 2.3 million synonyms
\item can be downloaded (as a Darwin Core Archive) for local use
\end{itemize}
\end{frame}


\begin{frame}{Darwin Core Archive}
	%\adjincludegraphics[height=.8\textheight,center]{dwca.jpg}    
	\adjincludegraphics[height=.9\textheight,center]{dwca.jpg}    
\end{frame}


\section{System design for the Guam Biodiversity Inventory}

\begin{frame}{System design for the Guam Biodiversity Inventory}
	\adjincludegraphics[height=.9\textheight,center]{diag1.pdf}     
\end{frame}


\end{document}
